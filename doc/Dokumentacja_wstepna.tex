\documentclass[a4size,final]{article}
\usepackage{fullpage}
\usepackage{polski}
\usepackage[cp1250]{inputenc}
\usepackage[polish]{babel}
%\usepackage{graphicx}
\usepackage[usenames,dvipsnames]{color}
\usepackage{hyperref}
\hypersetup{
    colorlinks,
    citecolor=black,
    filecolor=black,
    linkcolor=ForestGreen,
    urlcolor=blue
}
\begin{document}
\normalsize
\pagestyle{plain}
\setcounter{secnumdepth}{1}
\title{\Huge{Projekt ZPR}\\ \Large{Analiza widmowa sekwencji DNA}}
\author{Krzysztof Grzyb & Tomasz Zieli�ski}
\maketitle
\section{Opis projektu}
Celem projektu jest napisanie programu analizuj�cego sekwencje DNA i przedstawienie widma tych sekwencji w czytelnej formie.
\subsection{Wymagania funkcjonalne}
Program jest uruchamiany z linii polece�. Jako argument podawana jest nazwa pliku zawieraj�cego sekwencje DNA. Dla ka�dej z sekwencji DNA wykonywana jest transformata Fouriera.
\subsubsection{Format danych wej�ciowych}
Pliki z danymi s� zapisywane w dw�ch r�nych formatach z rozszerzeniem (jakim?). Parser sam rozpoznaje format pliku.
\subsubsection{Format danych wyj�ciowych}
W przypadku nierozpoznanego formatu zwracany jest komunikat o b��dzie, a program ko�czy dzia�anie.\\
Po wykonaniu tranformaty Fouriera wy�wietlana jest lista widm poszczeg�lnych sekwencji. U�ytkownik mo�e wybra� dowoln� z sekwencji i wy�wietli� jej widmo. Podawane s� podstawowe dane, takie jak d�ugo�� sekwencji (i inne?).
\section{Rozwi�zania programowe}
\subsection{Struktura programu}
\subsection{Styl kodowania}
\href{http://wiki.vcmi.eu/index.php?title=Coding_guidelines}{Proponuj� styl projektu VCMI}
\subsection{Wykorzystane biblioteki}
\begin{itemize}
\item GSL (FFT)
\item QT (wykres, ew. interfejs graficzny)
\item Boost (FOREACH, lexical\_cast, inne (Spirit?))
\end{itemize}
\section{Utrzymanie i rozszerzalno�� projektu}
\subsection{Przeno�no�� kodu}
\subsection{Mo�liwo�� obs�ugi dodatkowych format�w wej�ciowych}
\subsection{Plan realizacji (harmonogram prac)}
\begin{itemize}
\item Obs�uga bibliotek
\item Zapewnienie automatycznej kompilacji w srodowisku Linux i Windows
\item Parsowanie plik�w wej�ciowych
\item FFT
\item Interfejs graficzny
\item Termin dodatkowy
\end{itemize}
\end{document}
