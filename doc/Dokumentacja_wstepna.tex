\documentclass[a4size,final]{article}
\usepackage{fullpage}
\usepackage{polski}
\usepackage[utf8]{inputenc}
\usepackage[polish]{babel}
%\usepackage{graphicx}
\usepackage[usenames,dvipsnames]{color}
\usepackage{hyperref}
\hypersetup{
    colorlinks,
    citecolor=black,
    filecolor=black,
    linkcolor=ForestGreen,
    urlcolor=blue
}
\begin{document}
\normalsize
\pagestyle{plain}
\setcounter{secnumdepth}{1}
\title{\Huge{Projekt ZPR}\\ \Large{Analiza widmowa sekwencji DNA}}
\author{Krzysztof Grzyb & Tomasz Zieliński}
\maketitle


\section{Opis projektu}
\subsection*{Temat}
Obliczanie i wizualizacja widma transformaty Fouriera dla nietypowych danych np. sekwencji DNA. Sekwencję DNA możemy traktować jak napis a ten z kolei po zamianie na liczby jako ciąg próbek na podstawie których można policzyć transformatę.
\subsection{Wymagania funkcjonalne}
Program jest uruchamiany z linii poleceń. Jako argument podawana jest nazwa pliku zawierającego sekwencje DNA. Dla każdej z sekwencji DNA wykonywana jest transformata Fouriera.
\subsubsection{Format danych wejściowych}
Pliki z danymi są zapisywane w różnych formatach z rozszerzeniem .dat. Parser sam rozpoznaje format pliku.
\begin{itemize}
\item \textbf{Typ 1}\\
	Plik musi być typu tekstowego. Pierwszy wiersz zawiera jedną liczbę dodatnią, mówiącą na której pozycji następuje rozróżnienie między egzonem? a intronem. Następne wiersze występują w parach. Pierwszy mówi czy sekwencja DNA jest poprawna (1 poprawna 0 niepoprawna), a drugi zawiera sekwencję DNA.\\
Przykładowy plik z dwoma sekwencjami wygląda tak:\\ \\
\emph{7\\1\\CTCCGAAGTAGGATT\\1\\TCAGAAGGTGAGGGC}
\item \textbf{Typ 2}\\
Plik musi być typu tekstowego. Może składać się z wielu sekwencji DNA. Składnia sekwencji jest taka sama. Najpierw występuje etykieta, a w następnej lini dane. Każda sekwencja musi posiadać 4 etykiety które będą sprawdzane: Len: - długość sekwencji DNA, Introns - zakresy gdzie w sekwencji DNA znajdują się introny, Exons - zakresy gdzie w sekwencji DNA znajdują się Introny, Data - sekwencja dna\\
Przykładowy plik:\\
\emph{(nie wiem jak zapisać ten znak sprawdz plik tex)>Seq 0 Len:\\250\\5UTR\\\\Intergenic\\\\Introns\\ 10 55 100 130 150 170\\Exons\\ 56 99 180 200\\3UTR\\\\Data\\AAGCTTATTATCTCTCCTTGACTCTCATCCGAGCTATCTTCTTCCACAT\\
CTCTCTCGTTCCTCGGCGCGAACCTCTCGCTTCTTCTCCTCTTACTCCGATTGAACGATTCCGGATCT\\
CGCTCTTTCTTCCCCGGTCGATCTCTCACCGTCGCCACCTCCTGATGCATCTGCTCACGACTGTTTTC\\
TGAAACCCATGAACATAACAAAAACTCGTCAGAAAATAGAAACTAATCAAGGAAGAAGAAAAAGAAAA\\
GGAGAAGGAAGCAATACCTGAGATTCTCTGCCTCCGTTTAGTAACTCCGTTTGATGCAGCCATGTTTG\\
TCGTTGATCTTCGAGCTTCCCTGGGTTTATCCATTAATGCAAGCTAGAAAGACTAAGCTCCCATCAAC\\
TGAGTAAAATCAGATTCCACTCCTTCAAAGTTTTGAAGCTT}
\item \textbf{Typ 3}\\
Plik typu tekstowego. W pliku zapisane liczby pooddzielane pojedynczymi spacjami reprezentujące próbki sygnału, dla którego należy wyliczyć transformate Fouriera.
\end{itemize}

\subsubsection{Format danych wyjściowych}
W przypadku nierozpoznanego formatu zwracany jest komunikat o błędzie, a program kończy działanie.\\
Po wykonaniu tranformaty Fouriera wyświetlana jest lista widm poszczególnych sekwencji. Użytkownik może wybrać dowolną z sekwencji i wyświetlić jej widmo. Podawane są podstawowe dane, takie jak długość sekwencji (i inne?).
\section{Rozwiązania programowe}
\subsection{Struktura programu}
Program będzie składał się z trzech głównych modułów:
\begin{itemize}
\item Przetwarzanie plików wejściowych.\\
Moduł będzie wczytywał dane z pliku, rozpoznawał jakiego typu dane zostały wczytane i przetwarzał je do formatu używanego w programie, na którym możliwe jest zastosowanie FFT.
\item Obliczanie transformaty Fouriera.\\
\item Wizualizacja wyników.\\
Wyświetlanie wyników na ekranie w postaci wykresu i/lub zapis wyników do pliku.
\end{itemize}

\subsection{Styl kodowania}
\href{http://wiki.vcmi.eu/index.php?title=Coding_guidelines}{Proponuję styl projektu VCMI}
\subsection{Wykorzystane biblioteki}
\begin{itemize}
\item GSL (FFT)
\item QT (wykres, ew. interfejs graficzny)
\item Boost (FOREACH, lexical\_cast, inne (Spirit?))
\end{itemize}
\section{Utrzymanie i rozszerzalność projektu}
\subsection{Możliwość rozszerzania funkcjonalności}
Kod programu będzie można łatwo rozszerzać o nową funkcjonalność:
\begin{itemize}
\item Obsługiwanie dodatkowych formatów wejściowych
\item Inna wizualizacja wyników.
\end{itemize}
(możemy użyć wzorca dekoratora aby wyświetlać wyniki na wykresie i dodawać nowe opcje do wykresu np. linie poziome, pionowe, skalawanie wykresu ???)
\subsection{Przenośność kodu}
Kod będzie z założenia pisany multiplatformowo. Testowany na dwóch systemach Linux i Windows
\subsection{Plan realizacji (harmonogram prac)}
\begin{itemize}
\item Obsługa bibliotek
\item Zapewnienie automatycznej kompilacji w srodowisku Linux i Windows
\item Parsowanie plików wejściowych
\item FFT
\item Interfejs graficzny
\item Termin dodatkowy
\end{itemize}
\end{document}
